% DOCUMENT PREAMBLE

% In TeX, anything from a % sign to the end of a line is considered a
% "comment" and ignored in typesetting.

% The preamble describes the details of layout and typesetting. The
% description consists of a \documentclass declaration followed by a
% mixture of \usepackage{} commands and other commands. You can also
% define your own commands and macros, if you wish.

% Essentials. In LaTeX the first statement should always be
% \documentclass{}. Here we have chosen 12pt as the main font size and
% article as the "class" of the document. article and book are the main
% classes to care about.

\documentclass[12pt]{article}

% That's it for essentials. You could skip directly to DOCUMENT BODY.
% But I'll demonstrate how to adjust the page design a little.

% MULTILINGUAL SUPPORT
%
% Unlike XeLaTeX, LaTeX needs some extra add-ons to support multilingual
% text. There are actually many roads to multilingual LaTeX, but here
% is something relatively simple. With this you can enter accented
% characters as Unicode directly. Support for non-Western-European
% scripts requires further packages.

\usepackage[english]{babel}
\usepackage[utf8]{inputenc}

% Margins: use the geometry package.
% But N.B. standard word-processor margins really make your text
% body too narrow.

\usepackage{geometry} 
\geometry{width=6.0 in, height=8.5 in}

% FONTS
%
% Again unlike XeLaTeX, pdflatex uses TeX's native font system, which,
% unfortunately, is entirely disconnected from your system fonts. By
% default (pdf)latex will set your document in Knuth's Computer Modern
% font, which is perfectly functional but recognizably distinct from
% typical text fonts (it was designed to work well in documents with
% lots of equations and mathematical symbols). LaTeX ships with packages
% to use its own versions of some popular fonts: of these, I think
% Palatino is best.

\usepackage{palatino}

% If you desire, use double-spacing for a more open typescript.
% Again, not lovely from a design standpoint.
\usepackage{setspace}
\doublespacing

% The following incantation ensures that no section numbers will be
% typeset:

\setcounter{secnumdepth}{-2}

% The ``pagestyle'' controls things like headers and footers.
% ``plain'' style puts the page number at the center bottom.
\pagestyle{plain}

% END OF PREAMBLE

% DOCUMENT BODY
\begin{document}

\section{The basics of entering text}

In the body, ordinary text is simply typed in as you normally would. As in HTML, multiple spaces are converted      into  single spaces, and
a single carriage return is
ignored. You can space out your text in whatever way makes the most sense to you.

One or more blank lines in the source makes a paragraph break. 

If you wish to insert a linebreak of your own, \\ use a double backslash.

Typographical niceties: ``curly quotes'' double and `single'; the apostrophe's easy; and dashes---the em dash---and the en dash (for numbers, as in 1990--2000).

Here is an example of \emph{emphasis} and of \emph{emphasis with a \emph{further} emphasis within it}.

And the favorite humanist command.\footnote{It's the footnote, of course.}

\section{Remarks on document structure}

Note the way you notate document sections.

\subsection{Environments}

Documents are further structured by environments, which set text differently from ordinary paragraphs.
\begin{quote}
A blockquote.
\end{quote}
Ne vous inquiétez pas à propos des difficultés des accents et des autres langues. Au moins en ce qui concerne le français, l'allemand, l'espagnol, l'italien. 

Unicode aside, TeX has its own system for entering diacritics using commands: \'a \`a \"o \d s \u c.

% END OF DOCUMENT BODY
\end{document}  
% END OF DOCUMENT
% Nothing here will be typeset
